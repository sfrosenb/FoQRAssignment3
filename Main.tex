\documentclass[]{article}
\usepackage{lmodern}
\usepackage{amssymb,amsmath}
\usepackage{ifxetex,ifluatex}
\usepackage{fixltx2e} % provides \textsubscript
\ifnum 0\ifxetex 1\fi\ifluatex 1\fi=0 % if pdftex
  \usepackage[T1]{fontenc}
  \usepackage[utf8]{inputenc}
\else % if luatex or xelatex
  \ifxetex
    \usepackage{mathspec}
  \else
    \usepackage{fontspec}
  \fi
  \defaultfontfeatures{Ligatures=TeX,Scale=MatchLowercase}
\fi
% use upquote if available, for straight quotes in verbatim environments
\IfFileExists{upquote.sty}{\usepackage{upquote}}{}
% use microtype if available
\IfFileExists{microtype.sty}{%
\usepackage{microtype}
\UseMicrotypeSet[protrusion]{basicmath} % disable protrusion for tt fonts
}{}
\usepackage[margin=1in]{geometry}
\usepackage{hyperref}
\hypersetup{unicode=true,
            pdftitle={FoQR-HW-3},
            pdfauthor={Sam Rosenblatt},
            pdfborder={0 0 0},
            breaklinks=true}
\urlstyle{same}  % don't use monospace font for urls
\usepackage{color}
\usepackage{fancyvrb}
\newcommand{\VerbBar}{|}
\newcommand{\VERB}{\Verb[commandchars=\\\{\}]}
\DefineVerbatimEnvironment{Highlighting}{Verbatim}{commandchars=\\\{\}}
% Add ',fontsize=\small' for more characters per line
\usepackage{framed}
\definecolor{shadecolor}{RGB}{248,248,248}
\newenvironment{Shaded}{\begin{snugshade}}{\end{snugshade}}
\newcommand{\KeywordTok}[1]{\textcolor[rgb]{0.13,0.29,0.53}{\textbf{#1}}}
\newcommand{\DataTypeTok}[1]{\textcolor[rgb]{0.13,0.29,0.53}{#1}}
\newcommand{\DecValTok}[1]{\textcolor[rgb]{0.00,0.00,0.81}{#1}}
\newcommand{\BaseNTok}[1]{\textcolor[rgb]{0.00,0.00,0.81}{#1}}
\newcommand{\FloatTok}[1]{\textcolor[rgb]{0.00,0.00,0.81}{#1}}
\newcommand{\ConstantTok}[1]{\textcolor[rgb]{0.00,0.00,0.00}{#1}}
\newcommand{\CharTok}[1]{\textcolor[rgb]{0.31,0.60,0.02}{#1}}
\newcommand{\SpecialCharTok}[1]{\textcolor[rgb]{0.00,0.00,0.00}{#1}}
\newcommand{\StringTok}[1]{\textcolor[rgb]{0.31,0.60,0.02}{#1}}
\newcommand{\VerbatimStringTok}[1]{\textcolor[rgb]{0.31,0.60,0.02}{#1}}
\newcommand{\SpecialStringTok}[1]{\textcolor[rgb]{0.31,0.60,0.02}{#1}}
\newcommand{\ImportTok}[1]{#1}
\newcommand{\CommentTok}[1]{\textcolor[rgb]{0.56,0.35,0.01}{\textit{#1}}}
\newcommand{\DocumentationTok}[1]{\textcolor[rgb]{0.56,0.35,0.01}{\textbf{\textit{#1}}}}
\newcommand{\AnnotationTok}[1]{\textcolor[rgb]{0.56,0.35,0.01}{\textbf{\textit{#1}}}}
\newcommand{\CommentVarTok}[1]{\textcolor[rgb]{0.56,0.35,0.01}{\textbf{\textit{#1}}}}
\newcommand{\OtherTok}[1]{\textcolor[rgb]{0.56,0.35,0.01}{#1}}
\newcommand{\FunctionTok}[1]{\textcolor[rgb]{0.00,0.00,0.00}{#1}}
\newcommand{\VariableTok}[1]{\textcolor[rgb]{0.00,0.00,0.00}{#1}}
\newcommand{\ControlFlowTok}[1]{\textcolor[rgb]{0.13,0.29,0.53}{\textbf{#1}}}
\newcommand{\OperatorTok}[1]{\textcolor[rgb]{0.81,0.36,0.00}{\textbf{#1}}}
\newcommand{\BuiltInTok}[1]{#1}
\newcommand{\ExtensionTok}[1]{#1}
\newcommand{\PreprocessorTok}[1]{\textcolor[rgb]{0.56,0.35,0.01}{\textit{#1}}}
\newcommand{\AttributeTok}[1]{\textcolor[rgb]{0.77,0.63,0.00}{#1}}
\newcommand{\RegionMarkerTok}[1]{#1}
\newcommand{\InformationTok}[1]{\textcolor[rgb]{0.56,0.35,0.01}{\textbf{\textit{#1}}}}
\newcommand{\WarningTok}[1]{\textcolor[rgb]{0.56,0.35,0.01}{\textbf{\textit{#1}}}}
\newcommand{\AlertTok}[1]{\textcolor[rgb]{0.94,0.16,0.16}{#1}}
\newcommand{\ErrorTok}[1]{\textcolor[rgb]{0.64,0.00,0.00}{\textbf{#1}}}
\newcommand{\NormalTok}[1]{#1}
\usepackage{graphicx,grffile}
\makeatletter
\def\maxwidth{\ifdim\Gin@nat@width>\linewidth\linewidth\else\Gin@nat@width\fi}
\def\maxheight{\ifdim\Gin@nat@height>\textheight\textheight\else\Gin@nat@height\fi}
\makeatother
% Scale images if necessary, so that they will not overflow the page
% margins by default, and it is still possible to overwrite the defaults
% using explicit options in \includegraphics[width, height, ...]{}
\setkeys{Gin}{width=\maxwidth,height=\maxheight,keepaspectratio}
\IfFileExists{parskip.sty}{%
\usepackage{parskip}
}{% else
\setlength{\parindent}{0pt}
\setlength{\parskip}{6pt plus 2pt minus 1pt}
}
\setlength{\emergencystretch}{3em}  % prevent overfull lines
\providecommand{\tightlist}{%
  \setlength{\itemsep}{0pt}\setlength{\parskip}{0pt}}
\setcounter{secnumdepth}{0}
% Redefines (sub)paragraphs to behave more like sections
\ifx\paragraph\undefined\else
\let\oldparagraph\paragraph
\renewcommand{\paragraph}[1]{\oldparagraph{#1}\mbox{}}
\fi
\ifx\subparagraph\undefined\else
\let\oldsubparagraph\subparagraph
\renewcommand{\subparagraph}[1]{\oldsubparagraph{#1}\mbox{}}
\fi

%%% Use protect on footnotes to avoid problems with footnotes in titles
\let\rmarkdownfootnote\footnote%
\def\footnote{\protect\rmarkdownfootnote}

%%% Change title format to be more compact
\usepackage{titling}

% Create subtitle command for use in maketitle
\newcommand{\subtitle}[1]{
  \posttitle{
    \begin{center}\large#1\end{center}
    }
}

\setlength{\droptitle}{-2em}

  \title{FoQR-HW-3}
    \pretitle{\vspace{\droptitle}\centering\huge}
  \posttitle{\par}
    \author{Sam Rosenblatt}
    \preauthor{\centering\large\emph}
  \postauthor{\par}
      \predate{\centering\large\emph}
  \postdate{\par}
    \date{1/30/2019}


\begin{document}
\maketitle

First we load in the databases

\begin{Shaded}
\begin{Highlighting}[]
\KeywordTok{load}\NormalTok{(}\StringTok{"COMADRE_v.2.0.1.RData"}\NormalTok{)}
\KeywordTok{load}\NormalTok{(}\StringTok{"COMPADRE_v.4.0.1.RData"}\NormalTok{)}
\end{Highlighting}
\end{Shaded}

Brendan and I are working on a project modelling contagion in
populations of honey bee and bumble bee species where the contagion is
transmitted through several flowers.

Initially we tried to use this as an oppurtunity to investigate the
population models of the bees, however there are no models of bee
populations in the database, or even any Hymenoptera populations at all

\begin{Shaded}
\begin{Highlighting}[]
\KeywordTok{grep}\NormalTok{(comadre}\OperatorTok{$}\NormalTok{metadata}\OperatorTok{$}\NormalTok{Order,}\DataTypeTok{pattern =} \StringTok{'Hymenoptera'}\NormalTok{)}
\end{Highlighting}
\end{Shaded}

\begin{verbatim}
## integer(0)
\end{verbatim}

Thus, instead of working with the bee populations we decided to look at
one of the species of flower that is a transmission vector between
bumblebees and honeybees: The white clover, i.e. \emph{Trifolium repens}

First we check that it exists in the database:

\begin{Shaded}
\begin{Highlighting}[]
\KeywordTok{grep}\NormalTok{(compadre}\OperatorTok{$}\NormalTok{metadata}\OperatorTok{$}\NormalTok{SpeciesAccepted,}\DataTypeTok{pattern =} \StringTok{'Lotus corniculatus'}\NormalTok{)}
\end{Highlighting}
\end{Shaded}

\begin{verbatim}
##  [1] 3530 3531 3532 3533 3534 3535 3536 3537 3538 3539 3540 3541 3542 3543
## [15] 3544 3545 3546 3547
\end{verbatim}

From the above chunk we see that indeed it exists in the database, now
we would like to choose the model which is the most relevant to our
research, which is mainly concerned with these populations in Vermont
and the greater New England area. To see which study is most appropriate
for our purposes we will look at some of the info for each study.

\begin{Shaded}
\begin{Highlighting}[]
\NormalTok{allStudies =}\StringTok{ }\KeywordTok{grep}\NormalTok{(compadre}\OperatorTok{$}\NormalTok{metadata}\OperatorTok{$}\NormalTok{SpeciesAccepted,}\DataTypeTok{pattern =} \StringTok{'Lotus corniculatus'}\NormalTok{)}

\ControlFlowTok{for}\NormalTok{ (study }\ControlFlowTok{in}\NormalTok{ allStudies)\{}
  \KeywordTok{print}\NormalTok{(}\KeywordTok{cat}\NormalTok{(}\StringTok{"Continent: "}\NormalTok{, compadre}\OperatorTok{$}\NormalTok{metadata}\OperatorTok{$}\NormalTok{Continent[study]))}
  \KeywordTok{print}\NormalTok{(compadre}\OperatorTok{$}\NormalTok{metadata}\OperatorTok{$}\NormalTok{Country[study])}
  \KeywordTok{print}\NormalTok{(compadre}\OperatorTok{$}\NormalTok{metadata}\OperatorTok{$}\NormalTok{Ecoregion[study])}
  \KeywordTok{print}\NormalTok{(compadre}\OperatorTok{$}\NormalTok{metadata}\OperatorTok{$}\NormalTok{Lat[study])}
  \KeywordTok{print}\NormalTok{(compadre}\OperatorTok{$}\NormalTok{metadata}\OperatorTok{$}\NormalTok{Lon[study])}
  
\NormalTok{\}}
\end{Highlighting}
\end{Shaded}

\begin{verbatim}
## Continent:  5NULL
## [1] USA
## 61 Levels: ARG AUS AUT BEL BEN BOL BRA CAF CAN CHE CHL CHN CIV CMR ... ZAF
## [1] NA
## [1] NA
## [1] NA
## Continent:  5NULL
## [1] USA
## 61 Levels: ARG AUS AUT BEL BEN BOL BRA CAF CAN CHE CHL CHN CIV CMR ... ZAF
## [1] NA
## [1] NA
## [1] NA
## Continent:  5NULL
## [1] USA
## 61 Levels: ARG AUS AUT BEL BEN BOL BRA CAF CAN CHE CHL CHN CIV CMR ... ZAF
## [1] NA
## [1] NA
## [1] NA
## Continent:  5NULL
## [1] USA
## 61 Levels: ARG AUS AUT BEL BEN BOL BRA CAF CAN CHE CHL CHN CIV CMR ... ZAF
## [1] NA
## [1] NA
## [1] NA
## Continent:  5NULL
## [1] USA
## 61 Levels: ARG AUS AUT BEL BEN BOL BRA CAF CAN CHE CHL CHN CIV CMR ... ZAF
## [1] NA
## [1] NA
## [1] NA
## Continent:  5NULL
## [1] USA
## 61 Levels: ARG AUS AUT BEL BEN BOL BRA CAF CAN CHE CHL CHN CIV CMR ... ZAF
## [1] NA
## [1] NA
## [1] NA
## Continent:  5NULL
## [1] USA
## 61 Levels: ARG AUS AUT BEL BEN BOL BRA CAF CAN CHE CHL CHN CIV CMR ... ZAF
## [1] NA
## [1] NA
## [1] NA
## Continent:  5NULL
## [1] USA
## 61 Levels: ARG AUS AUT BEL BEN BOL BRA CAF CAN CHE CHL CHN CIV CMR ... ZAF
## [1] NA
## [1] NA
## [1] NA
## Continent:  5NULL
## [1] USA
## 61 Levels: ARG AUS AUT BEL BEN BOL BRA CAF CAN CHE CHL CHN CIV CMR ... ZAF
## [1] NA
## [1] NA
## [1] NA
## Continent:  5NULL
## [1] USA
## 61 Levels: ARG AUS AUT BEL BEN BOL BRA CAF CAN CHE CHL CHN CIV CMR ... ZAF
## [1] NA
## [1] NA
## [1] NA
## Continent:  5NULL
## [1] USA
## 61 Levels: ARG AUS AUT BEL BEN BOL BRA CAF CAN CHE CHL CHN CIV CMR ... ZAF
## [1] NA
## [1] NA
## [1] NA
## Continent:  5NULL
## [1] USA
## 61 Levels: ARG AUS AUT BEL BEN BOL BRA CAF CAN CHE CHL CHN CIV CMR ... ZAF
## [1] NA
## [1] NA
## [1] NA
## Continent:  5NULL
## [1] USA
## 61 Levels: ARG AUS AUT BEL BEN BOL BRA CAF CAN CHE CHL CHN CIV CMR ... ZAF
## [1] NA
## [1] NA
## [1] NA
## Continent:  5NULL
## [1] USA
## 61 Levels: ARG AUS AUT BEL BEN BOL BRA CAF CAN CHE CHL CHN CIV CMR ... ZAF
## [1] NA
## [1] NA
## [1] NA
## Continent:  5NULL
## [1] USA
## 61 Levels: ARG AUS AUT BEL BEN BOL BRA CAF CAN CHE CHL CHN CIV CMR ... ZAF
## [1] NA
## [1] NA
## [1] NA
## Continent:  5NULL
## [1] USA
## 61 Levels: ARG AUS AUT BEL BEN BOL BRA CAF CAN CHE CHL CHN CIV CMR ... ZAF
## [1] NA
## [1] NA
## [1] NA
## Continent:  5NULL
## [1] USA
## 61 Levels: ARG AUS AUT BEL BEN BOL BRA CAF CAN CHE CHL CHN CIV CMR ... ZAF
## [1] NA
## [1] NA
## [1] NA
## Continent:  5NULL
## [1] USA
## 61 Levels: ARG AUS AUT BEL BEN BOL BRA CAF CAN CHE CHL CHN CIV CMR ... ZAF
## [1] NA
## [1] NA
## [1] NA
\end{verbatim}

\begin{Shaded}
\begin{Highlighting}[]
\CommentTok{#compadre$metadata$Ecoregion[3530]}
\end{Highlighting}
\end{Shaded}


\end{document}
